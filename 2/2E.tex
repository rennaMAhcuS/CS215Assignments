\documentclass{article}
\usepackage{amsmath, amssymb, amsfonts}

\begin{document}

The original expression is:

\[
P_h[X = i] \approx \frac{\sqrt{2}}{\sqrt{\pi h} \left( 1 - \frac{i^2}{h^2} \right)^{\frac{h+1}{2}} \left( \frac{h + i}{h - i} \right)^{\frac{i}{2}}}
\]

We will approximate this expression for large $h$, aiming to reduce it to a Gaussian form.

\subsection*{Step 1: Approximation of $\left( 1 - \frac{i^2}{h^2} \right)^{\frac{h+1}{2}}$}

We begin by simplifying the term:

\[
\left( 1 - \frac{i^2}{h^2} \right)^{\frac{h+1}{2}}
\]

For large $h$, we assume that $\frac{i^2}{h^2}$ is small. We can then use the binomial expansion (or a Taylor series expansion) for small $x$, which states:

\[
(1 - x)^n \approx e^{-nx} \text{ for small } x
\]

Here, $x = \frac{i^2}{h^2}$ and $n = \frac{h+1}{2}$, so we approximate:

\[
\left( 1 - \frac{i^2}{h^2} \right)^{\frac{h+1}{2}} \approx e^{-\frac{h+1}{2} \cdot \frac{i^2}{h^2}}
\]

For large $h$, the factor $\frac{h+1}{2}$ is approximately $\frac{h}{2}$, so this simplifies further to:

\[
e^{-\frac{h}{2} \cdot \frac{i^2}{h^2}} = e^{-\frac{i^2}{2h}}
\]

Thus, the term $\left( 1 - \frac{i^2}{h^2} \right)^{\frac{h+1}{2}}$ is approximated by:

\[
e^{-\frac{i^2}{2h}}
\]

\subsection*{Step 2: Approximation of $\left( \frac{h+i}{h-i} \right)^{\frac{i}{2}}$}

Next, we simplify the term:

\[
\left( \frac{h+i}{h-i} \right)^{\frac{i}{2}}
\]

For large $h$, we can use the logarithmic approximation. First, express the ratio as:

\[
\frac{h+i}{h-i} = 1 + \frac{2i}{h-i}
\]

For large $h$, $h-i \approx h$, so we approximate the ratio as:

\[
\frac{h+i}{h-i} \approx 1 + \frac{2i}{h}
\]

Now, we apply the logarithmic expansion for small $x$, which states:

\[
\log(1 + x) \approx x \text{ for small } x
\]

Thus:

\[
\log\left( \frac{h+i}{h-i} \right) \approx \frac{2i}{h}
\]

Exponentiating both sides, we obtain:

\[
\left( \frac{h+i}{h-i} \right)^{\frac{i}{2}} \approx e^{\frac{i}{2} \cdot \frac{2i}{h}} = e^{\frac{i^2}{h}}
\]

Thus, the term $\left(\frac{h+i}{h-i}\right)^{\frac{i}{2}}$ is approximated by:

\[
e^{\frac{i^2}{h}}
\]

\subsection*{Step 3: Combining the Results}

Now we combine the results from Step 1 and Step 2. The original expression becomes:

\[
P_h[X = i] \approx \frac{\sqrt{2}}{\sqrt{\pi h}} e^{-\frac{i^2}{2h}} e^{\frac{i^2}{h}}
\]

Simplifying the exponents:

\[
P_h[X = i] \approx \frac{\sqrt{2}}{\sqrt{\pi h}} e^{-\frac{i^2}{2h}}
\]

This is a Gaussian distribution with mean 0 and variance proportional to $h$.

\end{document}
